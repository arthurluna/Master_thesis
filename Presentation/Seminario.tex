\documentclass[10pt]{beamer}

\usetheme[progressbar=frametitle]{metropolis}

\usepackage[absolute,overlay]{textpos}
\usepackage{booktabs}
\usepackage[scale=2]{ccicons}
%\usepackage[brazilian]{babel}
%\usepackage[T1]{fontenc}    % Selecao de codigos de fonte.
\usepackage[utf8]{inputenc} 

\usepackage{pgfplots}
\usepgfplotslibrary{dateplot}

\usepackage{xspace}
\newcommand{\themename}{\textbf{\textsc{metropolis}}\xspace}

\title{Aplicando a Teoria Mie-Debye para Caracterização de Parâmetros Físicos \\em Pinças Óticas}
\subtitle{Defesa de Dissertação de Mestrado}
\date{\today}
\author{Arthur Luna da Fonseca \\Orientadores: Paulo Américo Maia Neto, Rafael de Sousa Dutra}
\date{10 de Setembro de 2019}
\institute{Instituto de Física - Universidade Federal do Rio de Janeiro}
\titlegraphic{\hfill\includegraphics[height=2.cm]{../logo_ufrj}}

\begin{document}

\maketitle

%%%%%%%%%%%%%%%%%%%%%%%%%%%%%%%%%%%%%%%%%%%%%%%%%%%%%%%%%%%%%%%%%%%%%%%%%%%%%%%%%%%%%%%%%%%%%%%%%

\begin{frame}{Conteúdo}
  \setbeamertemplate{section in toc}[sections numbered]
  \tableofcontents[hideallsubsections]
\end{frame}

%%%%%%%%%%%%%%%%%%%%%%%%%%%%%%%%%%%%%%%%%%%%%%%%%%%%%%%%%%%%%%%%%%%%%%%%%%%%%%%%%%%%%%%%%%%%%%%%%

\section{Introdução}

%%%%%%%%%%%%%%%%%%%%%%%%%%%%%%%%%%%%%%%%%%%%%%%%%%%%%%%%%%%%%%%%%%%%%%%%%%%%%%%%%%%%%%%%%%%%%%%%%

\begin{frame}[fragile]{Introdução}
      \begin{center}
          \metroset{block=fill}
          \begin{exampleblock}{Objetivo}

		      \begin{itemize}

		        \item Caracterização do parâmetro de astigmatismo do sistema ótico através do experimento de torque ótico $\rightarrow$ microesferas com raio da ordem do comprimento de onda.

		        \item Caracterização do parâmetro de absorção da microesfera através da análise da posição de equilíbrio axial. $\rightarrow$ microesferas com raio maior que o comprimento de onda.

		      \end{itemize}

          \end{exampleblock}
      \end{center}
\end{frame}

%%%%%%%%%%%%%%%%%%%%%%%%%%%%%%%%%%%%%%%%%%%%%%%%%%%%%%%%%%%%%%%%%%%%%%%%%%%%%%%%%%%%%%%%%%%%%%%%%

\begin{frame}[fragile]{Introdução}

Introdução à teoria de pinças óticas: \\
    \begin{center}
        Modelagens da pinça ótica: interação de um campo (feixe) com o objeto espalhador (esfera pinçada).

        Campo: feixe fortemente focalizado.

        Objeto espalhador: simetria esférica.

        

    \end{center}

\end{frame}

\section{Regimes de tamanhos do objeto espalhador}

\begin{frame}[fragile]{Regimes de tamanhos do objeto espalhador}

Dois regimes distintos: \\
    \begin{center}
        Regime Rayleigh: raio da esfera muito menor que o comprimento de onda do feixe ($a\ll\lambda$).

        Aproximamos a esfera por um dipolo induzido:
        \begin{equation}
        {\mathbf F} = \frac{1}{2}\nabla({\mathbf p}\cdot{\mathbf E})=\frac{1}{2}\nabla(\alpha{\mathbf E}^2).
        \end{equation}

	      \begin{itemize}

		        \item Força aponta para região de maior intensidade do campo

		        \item Feixe fortemente focalizado : região de maior intensidade é o foco.

	      \end{itemize}
 

    \end{center}

\end{frame}

\begin{frame}[fragile]{Regimes de tamanhos do objeto espalhador}

    \begin{center}
        Regime de ótica geométrica: raio da esfera muito mairo que o comprimento de onda do feixe ($a\gg\lambda$).

        Ignorando efeitos de reflexão e de absorção, raios diametralmente\\opostos transferem momento para esfera na direção do foco.
        %
        \begin{picture}(320,250)
        \put(190,95){\includegraphics[scale=.40]{../geom_lateralIV}}
        \end{picture}

        \begin{picture}(320,250)
        \put(-17,332){\includegraphics[scale=.38]{../geom_foco_axialIII}}
        \end{picture}
        %
        
    \end{center}

\end{frame}

\begin{frame}[fragile]{Regimes de tamanhos do objeto espalhador}

    \begin{center}
        A posição de equilíbrio não necessariamente é no foco, \\pois a reflexão (a) transfere momento na direção normal ao plano tangente ao ponto de reflexão na esfera,\\enquanto a absorção (b) transfere na direção da propagação do raio:
        %
        \begin{picture}(320,250)
        \put(27,90){\includegraphics[scale=.25]{../reflex_absorII}}
        \end{picture}

    \end{center}

\end{frame}

\section{Introdução ao modelo MDSA+}

\begin{frame}[fragile]{Introdução ao modelo MDSA+}

    \begin{center}
    	Modelo Mie-Debye para a pinça ótica: modelo exato para o espalhamento por uma esfera de um campo fortemente focalizado.
        
        Campo expresso como uma superposição de ondas planas: vetores $\mathbf{k}(\theta_k,\phi_k)$ formam um cone sólido de meia abertura $\theta_0$:
        %
        \begin{equation}
        \mathbf{E}_{IN}=\int\limits_0^{\theta_0} d\theta_k \sin\theta_k \int\limits_0^{2\pi} d\phi_k\sqrt{\sin\theta_k} e^{i\mathbf{k}\cdot\mathbf{r}}\hat{\varepsilon}_k,
        \label{RichWolf}
        \end{equation}

        Termo $\sqrt{\sin\theta_k}$ vem da condição de seno de Abbe ($f\sin\theta_k=\rho$):

    \end{center}

\end{frame}

\begin{frame}[fragile]{Introdução ao modelo MDSA+}

    \begin{center}
        Para encontrarmos os campos espalhados pela esfera, calculamos os potenciais de Debye do campo incidente para cada caso de polarização circular, que são definidos por:
        %
        \begin{equation}
        \Pi^{E}=\sum\limits_{J} \Pi^{E}_J=\sum\limits_{J}\frac{ ({\mathbf r}\cdot{\mathbf E })_J }{J(J+1)} \qquad e\qquad \Pi^{M}=\sum\limits_{J} \Pi^{M}_J=\sum\limits_{J}\frac{ ({\mathbf r}\cdot{\mathbf H })_J }{J(J+1)} .
        \end{equation}

        O somatório em $J$ denota a expansão do campo em multipolos. 

        Cada multipolo do campo incidente será espalhado com uma amplitude, dada pelos coeficiente de Mie $a_j$ (para os multipolos elétricos) e $b_j$ (para o multipolo magnético).

    \end{center}

\end{frame}

\begin{frame}[fragile]{Introdução ao modelo MDSA+}

    \begin{center}
    	
        Para introduzirmos o efeito causado pelo perfil do feixe paraxial incidente na objetiva, inserimos a amplitude de campo em cada ponto da entrada da objetiva na equação \ref{RichWolf}:
        %
        \begin{equation}
        \mathbf{E}_{IN}=\int\limits_0^{\theta_0} d\theta_k \sin\theta_k \int\limits_0^{2\pi} d\phi_k\sqrt{\sin\theta_k} e^{i\mathbf{k}\cdot\mathbf{r}} e^{-\frac{f^2\sin^2\theta_k}{\omega_0^2}} \hat{\varepsilon}_k.        
        \end{equation}

        O caso de um feixe elipticamente polarizado é obtido superpondo ondas circularmente polarizadas:
        \begin{equation}
        \hat{\varepsilon}_k = a_{\sigma+}\hat{\bf{\epsilon}}_{\sigma+} + a_{\sigma-}\hat{\bf{\epsilon}}_{\sigma-} .     
        \end{equation}

    \end{center}

\end{frame}


\begin{frame}[fragile]{Introdução ao modelo MDSA+}

    \begin{center}

    	\hspace{1mm}\\\hspace{1mm}\\


        Depois que o feixe focalizado é produzido pela objetiva, ele incide no porta amostra, passando por uma interface de vidro e água. 
        %
        \begin{picture}(320,250)
        \put(45,90){\includegraphics[scale=.6]{../aberracao_esf4}}
        \end{picture}
    \end{center}

\end{frame}


\begin{frame}[fragile]{Introdução ao modelo MDSA+}

    \begin{center}
        Esse efeito insere uma fase em cada componente de onda plana do campo, de acordo com seu ângulo de incidência, que pode ser obtida através da lei de Snell:
        %
        \begin{equation}
        \Psi(z,\theta) = k\left( -\frac{L}{N}\cos\theta + N L\cos\theta_1 \right).
        \end{equation}

        Também haverá reflexão nessa interface, e portanto devemos levar a amplitude de transmissão de Fresnel em consideração:
        %
        \begin{equation}
        T(\theta)=\frac{2\cos\theta}{\cos\theta + N\cos\theta_1}.     
        \end{equation}
    \end{center}

\end{frame}

\begin{frame}[fragile]{Introdução ao modelo MDSA+}

    \begin{center}

    	\hspace{1mm}\\\hspace{1mm}\\
        Outras aberrações óticas também podem ser incluidas no modelo usado o formalismo de Zernike, que utiliza um conjunto de polinômios que formam uma base para descrever as fases de cada aberração ótica de um feixe. %Os polinômios de Zernike formam uma base ortonormal no círculo unitário, o que nos permite descrever qualquer aberração ótica do sistema.\\
        
        %Nos experimentos realizados pelo grupo, a aberração de interesse é o astigmatismo.

        \begin{picture}(320,250)
        \put(34,132){\includegraphics[scale=.4]{../feixe_astig}}
        \end{picture}
    \end{center}

\end{frame}

\begin{frame}[fragile]{Introdução ao modelo MDSA+}

    \begin{center}
        A fase associada ao astigmatismo é:
        %
        \begin{equation}
        \Phi_{astigmatismo}(\theta,\phi) = 2\pi A_{ast}\left( \frac{\sin\theta}{\sin\theta_0} \right)^2\cos2(\phi - \varphi_a),
        \end{equation}
        onde o círculo unitário foi definido como o círculo formado pela entrada da objetiva, que tem raio $R_p=f\sin\theta_0$.

        O parâmetro $A_{ast}$ é o chamado parâmetro de astigmatismo. 

        Um dos objetivos do trabalho é a determinação desse parâmetro por meio de ajustes do modelo.

    \end{center}

\end{frame}

\begin{frame}[fragile]{Introdução ao modelo MDSA+}

    \begin{center}
        Finalmente, expressamos a força em termos do fator de eficiência $\mathbf{Q}$:
        %
        \begin{equation}
        {\mathbf Q} = \frac{\left<{\mathbf F}\right>}{n_1 P/c}= |a_+|^2{\mathbf Q}^{(\sigma +)} + |a_-|^2{\mathbf Q}^{(\sigma -)} + {\mathbf Q}^{({\it cross},+-)} + {\mathbf Q}^{({\it cross},-+)}.
        \end{equation}
        
        onde $n_1$ é o índice de refração do meio em que a microesfera está inserida e $P$ é a potência do laser.
        \begin{itemize}

        	\item Fator de eficiência na direção $z$ ($Q_z$) $\rightarrow$ calcular a posição de equilíbrio axial nas simulações. 

        	\item Grandezas medidas experimentalmente $\rightarrow$ constante eláticas $\kappa_\rho$ e $\kappa_\phi$, definidas como:
        \end{itemize}

        \begin{equation}
        \kappa_\rho=-\frac{n_1 P}{c}\frac{\partial Q_\rho}{\partial \rho}\Big{|}_{\rho=0, \thinspace z=z_{eq}} \qquad, \qquad \kappa_\phi=\frac{n_1 P}{c}\frac{\partial Q_\phi}{\partial \rho}\Big{|}_{\rho=0, \thinspace z=z_{eq}}.
        \end{equation}
%
    \end{center}

\end{frame}

%%%%%%%%%%%%%%%%%%%%%%%%%%%%%%%%%%%%%%%%%%%%%%%%%%%%%%%%%%%%%%%%%%%%%%%%%%%%%%%%%%%%%%%%%%%%%%%%%

\section{Experimento de transferência de momento angular}

%%%%%%%%%%%%%%%%%%%%%%%%%%%%%%%%%%%%%%%%%%%%%%%%%%%%%%%%%%%%%%%%%%%%%%%%%%%%%%%%%%%%%%%%%%%%%%%%%

\begin{frame}[fragile]{Experimento de transferência de momento angular}
    \begin{center}
        Para entender a simulação, é necessário discutir o experimento que mede o torque ótico na microesfera. Começamos pela geração do feixe paraxial na mesa ótica, ilustrada a seguir:

        \begin{picture}(320,250)
        \put(34,90){\includegraphics[scale=.21]{../fig/setup}}
        \end{picture}

    \end{center}
\end{frame}

%%%%%%%%%%%%%%%%%%%%%%%%%%%%%%%%%%%%%%%%%%%%%%%%%%%%%%%%%%%%%%%%%%%%%%%%%%%%%%%%%%%%%%%%%%%%%%%%%

\begin{frame}[fragile]{Experimento de transferência de momento angular}
    \begin{center}
        Dessa forma, o feixe entra no microscópio com polarização definida e cintura bem determinada, parâmetros estes que são de grande importância para simulação.

        O feixe, então, é refletido por um espelho dicróico, passa pela objetiva e incide no porta amostra, onde há a solução de microesfera.

        Através do espelho dicróico podemos tomar as imagens da amostra com uma câmera (CMOS).

    \end{center}
\end{frame}


\begin{frame}[fragile]{Experimento de transferência de momento angular}
    \begin{center}

        \begin{picture}(310,250)
        \put(10,40){\includegraphics[scale=.35]{../fig/microscopio}}
        \end{picture}

        \begin{picture}(320,250)
        \put(185,350){\includegraphics[scale=.33]{../fig/QWP}}
        \end{picture}

    \end{center}
\end{frame}

\begin{frame}[fragile]{Experimento de transferência de momento angular}
    \begin{center}
        O procedimento do experimento começa encostando a microesfera na lamínula. Em seguida, abaixa-se o estágio popr uma distância bem determinada e o deslocamos na direção de $\hat{x}$ com velocidades bem definidas. A força de arrasto do líquido (força de Stokes) é proporcional à velocidade de deslocamento do fluido, e desloca a microesfera na mesma direção e sentido. 

        Assumimos que as forças que são realizadas na esfera pelo feixe ao ser deslocada do eixo em que $\rho=0$ são harmônicas. 

    \end{center}
\end{frame}

\begin{frame}

    \begin{center}
        Teremos, então, no equilíbrio de forças, $\vec{F_\phi}+\vec{F_\rho}=-\vec{F_s}$. Observando a projeção na direção $y$, onde $\vec{F_s}\cdot\hat{y}=0$ e $F_\rho\sin\phi = F_\phi\cos\phi$:
        %
        \begin{equation}
        \frac{F_\phi}{F_\rho} = \tan\phi \approx \phi,
        \end{equation}
        
        onde $\phi$ é a coordenada angular da popsição de equilíbrio. Como as forças óticas são lineares, temos:
        %
        \begin{equation}
        \frac{\kappa_\phi}{\kappa_\rho} \approx \phi.
        \end{equation}
%
    \end{center}

\end{frame}

\begin{frame}[fragile]{Experimento de transferência de momento angular}
    \begin{center}
        É importante destacar que o ângulo $\phi$ não varia com a velocidade. Podemos ver abaixo o diagrama de forças da microesfera deslocada do eixo, e a reta formada pelas posições de equilíbrio de várias velocidades do estágio diferentes.

        \begin{picture}(310,250)
        \put(-22,130){\includegraphics[scale=.2]{../fig/pos_eq_phi}}
        \end{picture}

        \begin{picture}(320,250)
        \put(145,385){\includegraphics[scale=.16]{../fig/forcas_estagio}}
        \end{picture}

    \end{center}
\end{frame}

%%%%%%%%%%%%%%%%%%%%%%%%%%%%%%%%%%%%%%%%%%%%%%%%%%%%%%%%%%%%%%%%%%%%%%%%%%%%%%%%%%%%%%%%%%%%%%%%%

\section{Simulação do experimento de transferência de momento angular}

\begin{frame}[fragile]{Simulação do experimento}

    \begin{center}

        Para iniciar a simulação, calcula-se a posição do foco do feixe em relação à interface, quando a microesfera está encostada na lamínula e em uma posição de equilíbrio estável, com ângulo $\psi$ da placa de quarto de onda (QWP) $\psi=0$ (polarização linear). Esta posição será chamada de $L_0$.

        A seguir, desloca-se o estágio do microscópio na direção $z$ por uma distância de $\delta z = 3 \mu m$. O feixe, portanto, estará a uma distância $L=L_0 + 3 \mu m \frac{n_1}{n}$ da interface vidro-água.

    \end{center}

\end{frame}

\begin{frame}[fragile]{Simulação do experimento}

    \begin{center}
        Uma vez que a posição do foco esteja determinada, calculamos a posição de equilíbrio para varias polarizações do feixe incidente diferentes, ou seja, variando $\psi$.

        \begin{picture}(310,250)
        \put(-20,100){\includegraphics[scale=.35]{../Qz_tipico}}
        \end{picture}

        \begin{picture}(320,250)
        \put(160,353){\includegraphics[scale=.36]{../zeq_psi_ast25II}}
        \end{picture}

    \end{center}

\end{frame}

%%%%%%%%%%%%%%%%%%%%%%%%%%%%%%%%%%%%%%%%%%%%%%%%%%%%%%%%%%%%%%%%%%%%%%%%%%%%%%%%%%%%%%%%%%%%%%%%%

\begin{frame}[fragile]{Resultados da simulação do experimento}

    \begin{center}
        Usando as posições de equilíbrio, calculamos as derivadas do fator de eficiência nas direções $\rho$ e $\phi$, e temos assim o ângulo da posição de equilíbrio.

        Variando o parâmetro de astigmatismo, usamos uma função erro para determinar qual conjunto de pontos numéricos tem maior verossimilhança com os dados experimentais. A função erro é definida como:

        \begin{equation}
        E(A_{ast})=\sum \limits_i \big{[} \phi_i^{exp} - \phi_i^{sim}(A_{ast}) \big{]}^2.
        \end{equation}

    \end{center}

\end{frame}

\begin{frame}[fragile]{Resultados da simulação do experimento}

    \begin{center}

        A seguir, dois exemplos de conjuntos de pontos teóricos com paprâmetros de astigmatismo diferentes ($A_{ast}=0$ na esquerda e $A_{ast}=0.36$ na direita):

        \begin{picture}(310,250)
        \put(-10,100){\includegraphics[scale=.38]{../Kphi_rho_Aast_dupla}}
        \end{picture}

    \end{center}

\end{frame}

\begin{frame}[fragile]{Resultados da simulação do experimento}

    \begin{center}

        O gráfico da função erro pelo parâmetro de astigmatismo apresenta um mínimo em $A_{ast}=0.24$, que corresponde ao valor mais provável do parâmetro de astigmatismo:

        \begin{picture}(310,250)
        \put(70,100){\includegraphics[scale=.38]{../erro_astigII}}
        \end{picture}

    \end{center}

\end{frame}

\begin{frame}[fragile]{Resultados da simulação do experimento}

    \begin{center}

        A seguir, mostramos o gráfico da posição angular $\phi$ em função do ângulo da placa de quarto de onda $\psi$ para o valor de $A_{ast}=0.24$:

        \begin{picture}(310,250)
        \put(70,100){\includegraphics[scale=.38]{../Kphi_rho_Aast024III}}
        \end{picture}

    \end{center}

\end{frame}

%%%%%%%%%%%%%%%%%%%%%%%%%%%%%%%%%%%%%%%%%%%%%%%%%%%%%%%%%%%%%%%%%%%%%%%%%%%%%%%%%%%%%%%%%%%%%%%%%

\section{Caracterização do parâmetro de absorção da microesfera}

%%%%%%%%%%%%%%%%%%%%%%%%%%%%%%%%%%%%%%%%%%%%%%%%%%%%%%%%%%%%%%%%%%%%%%%%%%%%%%%%%%%%%%%%%%%%%%%%%

\begin{frame}[fragile]{Caracterização do parâmetro de absorção da microesfera}

  \begin{center}
      O que chamamos de parâmetro de absorção da microesfera é a parte imaginára do seu índice de refração. 

      O poliestireno é um material amplamente usado em experimentos de pinças óticas, e seu parâmetro de absorção é fornecido na literatura como $Im(n_{poli})=0.002$.

      Simulações com esferas grandes usadas e aprisionadas em laboratório não apresentaram qualquer posição de equilíbrio axial para esse valor quando o feixe incidente possui astigmatismo. Essa foi a motivação para o estudo da caracterização dessa grandeza, juntamente com o fato de haver uma forte dependência do fator de eficiência na direção $z$ com ela.

  \end{center}

\end{frame}

%%%%%%%%%%%%%%%%%%%%%%%%%%%%%%%%%%%%%%%%%%%%%%%%%%%%%%%%%%%%%%%%%%%%%%%%%%%%%%%%%%%%%%%%%%%%%%%%%

\begin{frame}[fragile]{Resultados da simulação do experimento}

    \begin{center}

        Os maiores valores de $Im(n_{poli})$ que apresentaram uma posição próxima ao eixo $z=0$ foram para $Im(n_{poli})=0.0004$, o que poderia indicar um equilíbrio com a força peso. O comportamento do fator de eficiência $Q_z$ em função de $z$ para vários raios está apresentado no gráfico a seguir:

        \begin{picture}(310,250)
        \put(-15,100){\includegraphics[scale=.38]{../Qz_z_1e-34II}}
        \end{picture}

    \end{center}

\end{frame}

%%%%%%%%%%%%%%%%%%%%%%%%%%%%%%%%%%%%%%%%%%%%%%%%%%%%%%%%%%%%%%%%%%%%%%%%%%%%%%%%%%%%%%%%%%%%%%%%%

\begin{frame}[fragile]{Caracterização do parâmetro de absorção da microesfera}

  \begin{center}
      Para voltar a observar posições de equilíbrio, podemos levar o parâmetro de astigmatismo a 0 e procurar parâmetros de abosorção que apresentem uma posição de equilíbrio próxima ao foco. Para fazer isso, o cálculo numérico para esferas menores é mais rápido, e o resultado é equivalente. Dessa forma, os valores de $n_{poli}$ tais que a posição de equilíbrio esteja ao menos 3 raios de de distância do plano focal estão representados pela parte preenchida do gráfico:

      \begin{picture}(310,250)
      \put(70,115){\includegraphics[scale=.23]{../Regiao_de_pincamentoII}}
      \end{picture}

  \end{center}

\end{frame}

%%%%%%%%%%%%%%%%%%%%%%%%%%%%%%%%%%%%%%%%%%%%%%%%%%%%%%%%%%%%%%%%%%%%%%%%%%%%%%%%%%%%%%%%%%%%%%%%%

\begin{frame}[fragile]{Caracterização do parâmetro de absorção da microesfera}

  \begin{center}
      Podemos propor um experimento para tentar caracterizar esse parâmetro. Como a força em $z$ tem uma dependência forte com $Im(n_{poli})$, podemos testar a condição de pinçamento observando as flutuações de energia potencial da pinça e calculando o tamanho da berreira potencial em $z$ que a esfera deve ultrapassar para sair do aprisionamento.

      Definimos o potencial efetivo para microesfera confinada no eixo $z$:
      %
      \begin{equation}
      V(z)=-\int\limits_{z_0}^{z} \frac{n_1P}{c} Q_z dz + V(z_0).
      \end{equation}

      O tamanho da barreira potencial é, então:
      %
      \begin{equation}
      \Delta=\int\limits^{z_{inst}}_{z_{eq}} Q_z \frac{dz}{a}.
      \end{equation}

  \end{center}

\end{frame}


%%%%%%%%%%%%%%%%%%%%%%%%%%%%%%%%%%%%%%%%%%%%%%%%%%%%%%%%%%%%%%%%%%%%%%%%%%%%%%%%%%%%%%%%%%%%%%%%%

\begin{frame}[fragile]{Caracterização do parâmetro de absorção da microesfera}

  \begin{center}
      A seguir, ilustramos as grandezas discutidas anteriormente. Em laranja, o fator de eficiência $Q_z(z)$, e em azul o potencial $V(z)$:

      \begin{picture}(310,250)
      \put(60,100){\includegraphics[scale=.4]{../potencial_qz}}
      \end{picture}

  \end{center}

\end{frame}

%%%%%%%%%%%%%%%%%%%%%%%%%%%%%%%%%%%%%%%%%%%%%%%%%%%%%%%%%%%%%%%%%%%%%%%%%%%%%%%%%%%%%%%%%%%%%%%%%

\begin{frame}[fragile]{Caracterização do parâmetro de absorção da microesfera}

  \begin{center}
      Fialmente, calculamos o tamanho da barreira potencial $\Delta$ para vários valores de $Im(n_{poli})$:
      %
      \begin{picture}(310,250)
      \put(60,100){\includegraphics[scale=.4]{../Pot_EGII}}
      \end{picture}

  \end{center}

\end{frame}

%%%%%%%%%%%%%%%%%%%%%%%%%%%%%%%%%%%%%%%%%%%%%%%%%%%%%%%%%%%%%%%%%%%%%%%%%%%%%%%%%%%%%%%%%%%%%%%%%

\begin{frame}[fragile]{Caracterização do parâmetro de absorção da microesfera}

  \begin{center}

      Multiplicamos o valor de $\Delta$ por $(n_1 P a)/c$ para obter o valor da barreira potencial de fato.

      Podemos medir os valores máximos de flutuações de energia potencial da microesfera para tempos curtos, calibrando a constante elástica experimentalmente e observando as amplitudes de oscilação da microesfera. Ao diminuir a potencial $P$ do laser, podemos procurar o valor em que a microesfera escapa da pinça pelo eixo $z$.

      Medimos, então, a energia potencial máxima que a microesfera pode obter e a potencia do laser que permite a microesfera escapar. Portanto, podemos procurar o valor de $Im(n_{poli})$ que fornece uma barreira menor ou igual à energia da microesfera
 
  \end{center}

\end{frame}

%%%%%%%%%%%%%%%%%%%%%%%%%%%%%%%%%%%%%%%%%%%%%%%%%%%%%%%%%%%%%%%%%%%%%%%%%%%%%%%%%%%%%%%%%%%%%%%%%


%\begin{frame}{Animation}
%  \begin{itemize}[<+- | alert@+>]
%    \item \alert<4>{This is\only<4>{ really} important}
%    \item Now this
%    \item And now this
%  \end{itemize}
%\end{frame}



%{%
%\setbeamertemplate{frame footer}{---->Put footer in HERE}
%\begin{frame}[fragile]{Frame footer}
%    \themename defines a custom beamer template to add a text to the footer. It can be set via
%    \begin{verbatim}\setbeamertemplate{frame footer}{My custom footer}\end{verbatim}
%\end{frame}
%}

\section{Conclusão}

\begin{frame}[fragile]{Conclusão}

  \begin{center}

      \begin{itemize}

        \item bla


      \end{itemize}
 
  \end{center}

\end{frame}

%\begin{frame}[standout]
%  Questions?
%\end{frame}

\appendix


\end{document}
