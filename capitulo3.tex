\begin{chapter}{Experimento e Simula��o}
\label{cap3}

\hspace{5 mm}Neste cap�tulo vamos apresentar o experimento que foi simulado com a teoria MDSA+ para obter os resultados da presente disserta��o, bem como os procedimentos de tal simula��o. 

\section{M�todo Experimental.}

\hspace{5 mm} O aparato que descreverei nessa se��o mede a transfer�ncia de momento angular da luz na pin�a �tica. O experimento foi realizado em 2018 por Diniz {\it et al.} \cite{Diniz2019}, e foi tamb�m resultado do mestrado do mesmo \cite{Diniz}. Detalhes do procedimento experimental, do material e das ferramentas usadas nesse experimento podem ser encontradas em sua tese e no artigo citado, e n�o ser�oa abordados nesse trabalho.

Podemos dividir a descri��o do experimento em {\bf dizer numero de partes} partes. Primeiramente, trataremos do feixe paraxial: trata-se de um laser com comprimento de onda $\lambda_0=1064nm$. Suas carater�sticas s�o importantes para definir o feixe focalizado pela objetiva, como vimos no cap�tulo anterior. A largura da cintura $\omega_0$ do feixe � essencial para garantir a condi��o de aprisionamento da microesfera, pois se esta for muito menor que a entrada da objetiva, teremos o que se chama de {\it underfilling}, que faz com que o fator de efici�ncia da pin�a seja muito reduzida. Por outro lado, {\bf conferir isso com paulo ou rafael} preencher a entrada da objetiva pode fazer com que a meia abertura do feixe focalizado alcance �ngulos que produziriam reflex�o total na interface entre a lam�nula (vidro) e a �gua {\bf falar antes da solu��o? aberra��o esf�rica foi tratada no cap anterior}. Isso significa que ondas evanescentes s�o produzidas, e podem alterar a for�a �tica quando a microesfera estiver perto da lam�nula.



\end{chapter}
