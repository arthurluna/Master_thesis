\begin{chapter}{Teoria da Pin�a �tica}
\label{cap2}

%\hspace{5 mm} BLABLABLA

\section{Introdu��o}

Nessa sec��o, discutirei o modelo Mie-Debye para o experimento de pin�a �tica, usado para obter os resultados da presente disserta��o. Os primeiros modelos que tentam descrever as for�as da pin�a �tica fazem uso de diversas aproxima��es para descrever o feixe que passa pela objetiva e a intera��o da esfera com o campo. Esse modelo, por outro lado, descreve o feixe de forma exata, de acordo com o formalismo de Richards-Wolf para um feixe fortemente focalizado, que leva em conta diversos efeitos que s�o ignorados pelos demais, al�m de ser v�lido para um espectro maior de rela��es entre o comprimento de onda $\lambda$ e o raio $a$.




\section{Espalhamento}

\section{Intera��o Spin-�rbita.}
blablabla...

blablabla...

\end{chapter}
