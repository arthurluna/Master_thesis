\begin{chapter}{Teoria da Pin�a �tica}
\label{cap2}

%\hspace{5 mm} BLABLABLA

\section{Introdu��o}

Nessa sec��o, discutirei brevemente a evolu��o dos modelos para o experimento de pin�a �tica, dando �nfase, em partiular, ao modelo Mie-Debye, usado para obter os resultados da presente disserta��o. Os primeiros modelos que tentam descrever as for�as da pin�a �tica fazem uso de diversas aproxima��es para descrever o feixe que passa pela objetiva e a intera��o da esfera com o campo. Um um dos primeiros modelos, proposto por A. Ashkin, o campo seria descrito por meio de �tica de raios, limite que n�o descreve casos em que o comprimento de onda � da ordem ou maior que o raio da esfera aprisionada. 




\section{Espalhamento}

\section{Intera��o Spin-�rbita.}
blablabla...

blablabla...

\end{chapter}
